\documentclass[12pt]{article}
\title{Division $*$-algebras}
\author{James Moody}
\date{April 2020}
\usepackage{amsmath}
\usepackage{amsfonts}
\usepackage{amssymb}
\usepackage{amsthm}
\usepackage{bbm}
\usepackage[margin=0.5in]{geometry}

\newtheorem{theorem}{Theorem}[section]
\newtheorem{lemma}[theorem]{Lemma}
\newtheorem{conjecture}[theorem]{Conjecture}
\newtheorem{condition}[theorem]{Condition}
\newtheorem{claim}[theorem]{Claim}
\newtheorem{question}[theorem]{Question}
\newtheorem{corollary}[theorem]{Corollary} 
\newtheorem{observation}[theorem]{Observation} 
\theoremstyle{definition}
\newtheorem{definition}[theorem]{Definition}
\newtheorem{statement}[theorem]{Statement}
\newtheorem{notation}[theorem]{Notation} 
\theoremstyle{remark}
\newtheorem{remark}[theorem]{Remark}


\begin{document}

\maketitle

\section*{Introduction} 

\begin{definition} A $*$-algebra over a field $\mathcal{K}$ is a vector space $\mathcal{A}$ over $\mathcal{K}$ together with a not necessarily commutive nor associative binary multiplication operator $\cdot: \mathcal{A} \times \mathcal{A} \rightarrow \mathcal{A}$, and a unary involution $^*:\mathcal{A} \rightarrow \mathcal{A}$ satisfying the following axioms:

(1) Left distributivity of multiplication over addition: $\mathbf{x} \cdot (\mathbf{y} + \mathbf{z}) = (\mathbf{x} \cdot \mathbf{y}) + (\mathbf{x} \cdot \mathbf{z})$

(2) Right distributivity of multiplication over addition: $(\mathbf{x}+\mathbf{y}) \cdot \mathbf{z} = (\mathbf{x} \cdot \mathbf{z}) + (\mathbf{y} \cdot \mathbf{z})$

(3) Compatibility with scalars: $(a\mathbf{x}) \cdot (b \mathbf{y}) = (ab)(\mathbf{x} \cdot \mathbf{y})$

(4) $^*$ is an additive homomorphism: $(\mathbf{x}+\mathbf{y})^* = \mathbf{x}^* + \mathbf{y}^*$

(5) $^*$ is a multiplicative anti-homomorphism: $(\mathbf{x} \cdot \mathbf{y})^* = \mathbf{y}^* \cdot \mathbf{x}^*$
\end{definition}

\begin{definition} A \textbf{presentation} of a $*$-algebra over a field $\mathcal{K}$ is a vector space $V$ over \(\mathcal{K}\) together with a $\mathcal{K}$-linear function $L: V \rightarrow End(V)$ and an involution $^{\dagger} : V \rightarrow V$ such that for all \(x,y \in V\) $(L(x)y)^{\dagger} = L(y^{\dagger}) x^{\dagger}$.\\

\noindent We say \((V, L, {}^\dagger)\) is a \textbf{presentation of \(\mathcal{A}\)} if there is a linear isomorphism \(\pi: \mathcal{A} \rightarrow V\) such that:

(a) For all $\mathbf{x}, \mathbf{y} \in \mathcal{A}$, $\pi(\mathbf{x} \cdot \mathbf{y}) = L(\pi(\mathbf{x})) \pi(\mathbf{y})$ 

(b) For all $\mathbf{x} \in \mathcal{A}$, $\pi(\mathbf{x}^*) = \pi(\mathbf{x})^{\dagger}$
\end{definition} 

\begin{observation} Every \(*\)-algebra \(\mathcal{A}\) over a field \(\mathcal{K}\) has a presentation where \(\pi\) is the forgetful map from \(\mathcal{A}\) to the underling vector space \(V\) of \(\mathcal{A}\), \(L(x)y := \pi(\pi^{-1}(x) \cdot \pi^{-1}(y)\)), and \(x^{\dagger} := \pi(\pi^{-1}(x)^*)\).
\end{observation}

\begin{proof}
Properties (a) and (b) are satisfied by definition. It's clear \(\dagger\) is an involution. \((x,y) \mapsto L(x)y\) is bilinear since it is the pushforward of multiplication in \(\mathcal{A}\), so \(x \mapsto L(x)\) is \(\mathcal{K}\)-linear. It suffices to show then that for all \(x,y \in V\), $(L(x)y)^{\dagger} = L(y^{\dagger}) x^{\dagger}$. We can see this as follows:

\[(L(x)y)^{\dagger} = \pi(\pi^{-1}(L(x)y)^*) = \pi((\pi^{-1}(x) \cdot \pi^{-1}(y))^*)\] \[ = \pi(\pi^{-1}(y)^* \cdot \pi^{-1}(x)^*)= \pi(\pi^{-1}(y^{\dagger}) \cdot \pi^{-1}(x^{\dagger})) = L(y^{\dagger})x^{\dagger}\].
\end{proof}



\begin{definition}
A division $*$-algebra over $\mathcal{K}$ is a $*$-algebra over $\mathcal{K}$ satisfying the following two additional conditions:

(6) Right division: for any $\mathbf{x}$ and any non-zero $\mathbf{y}$, there exists a unique $\mathbf{z}$ such that $\mathbf{x} = \mathbf{z} \cdot \mathbf{y}$

(7) Left division: for any $\mathbf{x}$ and any non-zero $\mathbf{y}$, there exists a unique $\mathbf{z}$ such that $\mathbf{x} = \mathbf{y} \cdot \mathbf{z}$
\end{definition}
\end{document}